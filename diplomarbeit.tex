%% Dokumentklasse KOMA-Script Report
\documentclass[paper=a4, 12pt]{scrreprt}
%% Encoding UTF8
\usepackage[utf8]{inputenc}
%% 8 Bit Aufloesung der Buchstaben
\usepackage[T1]{fontenc}
%% Seitenraender
\usepackage[scale=0.85]{geometry}
%% Spracheinstellungen
\usepackage[english, naustrian]{babel} % your native language must be the last one!!
%% erweiterte Farbenpalette
\usepackage[dvipsnames]{xcolor}
%% Abbildungen
\usepackage{float}
\usepackage{adjustbox}
%% Tabellen (erweitert)
\usepackage{tabularx}
%% TikZ + Circuit-TikZ (fuer Schaltungen)
\usepackage[europeanresistors, europeaninductors]{circuitikz}
%% Nuetzliche TikZ Libraries
\usetikzlibrary{arrows, automata, positioning}
%% mathematik
\usepackage{amsmath, amssymb}
%\usepackage{mathtools}	
%% pdf-einbindung
\usepackage{pdfpages}
%% scource-code einbindung
\usepackage{listings, scrhack} %scrhack vermeidet Umschaltung auf KOMA Floats..
\usepackage{courier}
%% euro-symbol
\usepackage{eurosym}
%% landcsape-seiten ermöglichen
\usepackage{lscape}

%% Diplomarbeits-Format
\usepackage{srdpdipa}

%% Abkuerzungsverzeichnis
\usepackage[]{acronym}

%% Todos
\usepackage[]{todonotes}

%% Ganttdiagramme
\usepackage{pgfgantt}

%% Subfigures
\usepackage[lofdepth]{subfig}

%% definitionen =====================================%%
\dataSchool{HTBLuVA St. Pölten}
\dataDepartment{Höhere Lehranstalt für Elektronik und Technische Informatik}
\dataSubdepartment{Ausbildungsschwerpunkte Embedded Systems}
\dataSession{2019/20}

\title{CAN-Bus gesteuerte Stromquelle}
\author{Florian Hintermeier \and Dominik Gansch}
\date{\today}
\place{St. P\"olten}
\professor{Dipl.-Ing. Josef Radlbauer}
%%====================================================%%

% Hyperlinks im Dokument
\usepackage[colorlinks=true,
    linkcolor=black,
    citecolor=green,
    bookmarks=true,
    urlcolor=black,
    bookmarksopen=true]{hyperref}

\begin{document}

\frontmatter

%% titelseite ==========================================%%
\maketitle
%%======================================================%%

%% komplett leere seite ================================%%
\newpage\null\thispagestyle{empty}\newpage
%%======================================================%%

%% eidesstattliche erklärung ===========================%%
%%\begin{affidavit}
%%    \unterschrift{Florian Hintermeier}
%%    \unterschrift{Dominik Gansch}
%%\end{affidavit}
%%======================================================%%

%% dokumentation (deutsch/englisch) ====================%%
%%\includepdf[pages=-]{form/dokumentation-de.pdf}
%%\includepdf[pages=-]{form/dokumentation-en.pdf}
%%======================================================%%

%% inhaltsverzeichnis ==================================%%
\tableofcontents
%%======================================================%%

%% HAUPTTEIL ===========================================%%
\responsible{Florian Hintermeier, Dominik Gansch}
\mainmatter

\chapter{Einleitung}
    Die Firma ZKW möchte für zahlreiche in Entwicklung und bereits in Produktion befindliche LED Einheiten einen universell einsetzbaren LED Treiber mit linear einstellbarem Strom. Dabei soll jede einzelne LED von maximal 84 individuell in ihrer Helligkeit eingestellt werden können.
    
	\begin{adjustbox}{center,caption={Blockschaltbild von ZKW},label={somelabel},nofloat=figure,vspace=\bigskipamount}
	\includegraphics[width=\textwidth]{img/ZKW_Blockschaltbild.PNG}
	\end{adjustbox}

	Das Ziel liegt darin, eine geeignete Schaltung zu entwickeln um den von ZKW gegebenen Anforderungen zu entsprechen. Zu sehen sind diese in Abbildung 1.1.

\chapter{Individuelle Zielsetzung}
    \section{Hardware}    	
    	\subsubsection{Analogteil}
        Dieser Teil wird von Florian Hintermeier entwickelt. Er enthält die Schaltung um die Angeforderten Spannungen und Ströme erzeugen zu können, die von einem Mikrocontroller eingestellt werden sollen.
        
        \subsubsection{Digitalteil}
        Dieser Teil wird von Dominik Gansch entwickelt. Er beinhaltet den Mikrocontroller, der von der ZKW application CAN befehle entgegennimmt und diese richtig umsetzt und die richtigen Einstellungen am Analogteil der Schaltung trifft.
    
    %%\section{Software}

\chapter{Grundlagen und Methoden}
	\section{Verwendete Software}
	Die hier aufgeführte und kurz erklärte Software wurde für die Erreichung der Ziele in diesem Projekt verwendet.
	
	\paragraph{LTSpice}
	\paragraph{Draw.io}
	\paragraph{Altium Designer 19}
	\paragraph{TeXstudio}
	
	\section{Berechnungen}
	
\chapter{Ergebnisse}
	\section{Analogteil}
		\subsection{Schaltung}
		\subsection{Simulationen}
	
	\section{Digitalteil}
		\subsection{Schaltung}


%% ANHANG ==============================================%%
\appendix

%% abkürzungsverzeichnis ===============================%%
%% start of file abkuerzungen.tex

% Abkuerzungsverzeichnis
\addchap{
	\iflanguage{english}{Acronyms}{Abkürzungsverzeichnis}}
\begin{acronym}[ACRONYM]
\acro{adc}[ADC]{Analog Digital Converter}
\acro{can}[CAN]{Controller Area Network}
\acro{dac}[DAC]{Digital Analog Converter}
\acro{ldo}[LDO]{Low-Dropout regulator}
\acro{lmm}[LMM]{LED Matrix Manager}
\acro{pcb}[PCB]{Printed Circuit Board}
\acro{sepic}[SEPIC]{Single Ended Primary Inductance Converter}
\acro{opv}[OPV]{Operationsverstärker}
\acro{fet}[FET]{Field-Effect Transistor}
\acro{led}[LED]{Light Emitting Diode}
\acro{z-diode}[Z-Diode]{Zener Diode}
\acro{r}[R]{Resistor}
\acro{c}[C]{Capacitor}
\acro{gper}[GPER]{General Purpose Enable Register}
\acro{canif}[CANIF]{Control Area Network Interface}
\acro{cancfg}[CANCFG]{Control Area Network Configuration}
\acro{canctrl}[CANCTRL]{Control Area Network Controll}
\acro{mbits}[Mbit/s]{Megabit pro Sekunde}
\acro{mosi}[MOSI]{Master Out Slave In}
\acro{miso}[MISO]{Master In Slave Out}
\acro{sda}[SDA]{Serial Data Out}
\acro{sdi}[SDI]{Serial Data In}
\acro{ucd}[UCD]{User Controlled Device}
\acro{led}[LED]{Light Emitting Diode}
\acro{mosfet}[MOSFET]{Metal–Oxide–semiconductor Field-Fffect Transistor}
\acro{fet}[FET]{Field-Fffect Transistor}
\acro{gpio}[GPIO]{General Purpose In Out}
\acro{i/o}[I/O]{In/Out}
\acro{tdr}[TDR]{Transmit Data Register}
\end{acronym}\newpage

%% end of file abkuerzungen.tex
%%======================================================%%

%% abbildungsverzeichnis ===============================%%
\setcounter{lofdepth}{2}
\dipalistoffigures
%%======================================================%%

%% tabellenverzeichnis =================================%%
\setcounter{lotdepth}{2}
\dipalistoftables
%%======================================================%%

%% danksagungen=========================================%%
%%\begin{acknowledgements}
	\begin{center}
    Wir danken allen die die uns mit Ratschlägen und Ideen bei unserer Arbeit geholfen haben. Besonderer Dank fällt an Dipl.-Ing. Josef Radlbauer, der uns als Betreuer unserer Diplomarbeit immer wieder mit Ratschlägen und Ideen zur Seite Stand. Wir danken auch Dipl.-Ing. Markus Tillich für Tipps und Ratschläge beim Schaltungsdesign des Schaltwandlers. Besonderer Dank fällt auch an Ing. Rudolf Janeczek BEd MSc, der für uns den Kontakt mit der Firma ZKW aufgebaut hat, die uns mit der Idee für die Diplomarbeit betraut machten. Den Vertreten der Firma ZKW, Matthäus Artmann und Daniel Seitl, welche unsere Hauptansprechpartner waren, möchten wir besonders danken. 
	\end{center}
\end{acknowledgements}
%% Für Diplomarbeit hinzufügen
%%======================================================%%

%% literaturverzeichnis ================================%%
%%\newewpage
%%\input{tex/literatur.tex}
%%======================================================%%

%% betreuungsprotokolle ================================%%
%%\includepdf[pages=-]{form/betreuungsprotokoll_1.pdf} Noch hinzufügen
%% =====================================================%%

\end{document}
